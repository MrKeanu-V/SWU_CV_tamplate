\include{settings}

%%%%%%%%%%%%%%%%%%%%%%%%%%%%%%%
%%%%%%%%%  页眉页脚  %%%%%%%%%%%
%%%%%%%%%%%%%%%%%%%%%%%%%%%%%%%
% 定义填写学院信息
\newcommand{\school}{人工智能学院 | College of AI} 

% 定义联系方式
\newcommand{\contact}{
    % 根据个人喜好选择字号
    % \small              % 小
    % \footnotesize       % 更小
    \scriptsize         % 再小一号
    \textcolor{white}{
        % 邮箱 \href中写动态链接 后面写为文本,支持点击文本自动跳转
        \faEnvelope \quad \href{mailto:warmthwind@163.com}{warmthwind@163.com}
        \hspace{4em}
        % GitHub
        \faGithub \quad \href{https://github.com/MrKeanu-V}{MrKeanu-V}
        \hspace{4em}
        % 手机号
        \faPhone \quad  187****4104(微信同号)
        % 添加其他内容如微信、GitHub等格式如下,使用\fa命令添加不同icon,细节请搜索Font Awesome图标库
        % \hspace{4em}
        % \faGithub \quad \href{https://github.com/xxxx}{https://github.com/xxxx}
    }
}


%%%%%%%%%%%%%%%%%%%%%%%%%%%%%%%
%%%%%%%%%  简历正文  %%%%%%%%%%%
% 写简历还有项目细节时不要泛泛而谈太笼统,要应用STAR原则,即Situation(情景)、Task(任务)、Action(行动)和Result(结果)四个英文单词的首字母组合。
% S指的是situation,事情是在什么情况下发生
% T指的是task,你是如何明确你的目标的
% A指的是action,针对这样的情况分析,你采用了什么行动方式
% R指的是result,结果怎样,在这样的情况下你学习到了什么
%%%%%%%%%%%%%%%%%%%%%%%%%%%%%%%
\begin{document}
% 如果有多页简历,请把页眉页脚和背景复制粘贴到第二页的内容之前
   %%%%%%%%%  1、页眉页脚  %%%%%%%%%%%
    % 页眉:校标组合+学院名
    \begin{tikzpicture}[remember picture, overlay]
        \node[anchor=north, inner sep=0pt](header) at (current page.north){
            \includegraphics[width=\paperwidth]{images/header.png}
        };
        \node[anchor=west](school_logo) at (header.west){
            \hspace{0.5cm}
            \includegraphics[width=0.15\textwidth]{images/swu_logo_1_AIenhanced.png}
        };
        \node[anchor=east](school_name) at(header.east){
            \textcolor{white}{\textbf{\school}}
            \hspace{0.5cm}
        };
    \end{tikzpicture}
    \vspace{-3.5em}

    % 页脚:为文首声明的\contact,
    \begin{tikzpicture}[remember picture, overlay]
        \node[anchor=south, inner sep=0pt](footer) at (current page.south){
            \includegraphics[width=\paperwidth]{images/footer.png}
        };
        % 需要更改时直接更改文首的页眉页脚即可
        \node[anchor=center] at(footer.center){\contact};
    \end{tikzpicture}

    % 背景:校徽背景
    \begin{tikzpicture}[remember picture, overlay]
        \node[opacity=0.05] at(current page.center){
            \includegraphics[width=0.7\paperwidth, keepaspectratio]{images/swu_logo_big_blue.png}
        };
    \end{tikzpicture}


    %%%%%%%%%  2、简历正文  %%%%%%%%%%%
    % 个人信息
    \begin{figure}[h]
        % 左半边,信息,比例占行宽75%,可以自己调,非必要不建议调整
        \begin{minipage}{0.75\textwidth}
            \section{\makebox[\widthof{\faAddressCard}][cc]{\color{SWU_Blue}{\faAddressCard}}\quad 个人信息}
            \begin{tabularx}{\linewidth}{p{\widthof{出生日期:}}Xp{\widthof{政治面貌:}}X}
                姓\ \ \ \ \ \ \ \ \ 名: & 方鸿渐 & 
                性\ \ \ \ \ \ \ \ \ 别: & 男  \\
                出生年月: & XXXX.XX.XX & 
                籍\ \ \ \ \ \ \ \ \ 贯: & 重庆  \\
                %% 想多加几行的话,就按上面的格式自行补充
                %% 想加粗的话\textbf{}
                %% 想多加几列的话,把\begin{tabularx}{\textwidth}{这里}的内容改一下,可以自己搜一下tabularx怎么用,也可以问gpt/文心一言/讯飞。
            \end{tabularx}
        \end{minipage}
    \hspace{2em}
    % 右半边,照片,比例占行宽12%,可以自己调
    % images/avatar.png 替换成你证件照的路径。
    \begin{minipage}{0.16\textwidth}
        \setlength{\fboxsep}{0pt}
        \fcolorbox{SWU_Blue}{white}{\includegraphics[width=\linewidth]{images/my_avatar.png}}
        % \fcolorbox{\includegraphics[width=\linewidth]{images/my_avatar.png}}
        % \fbox{}
    \end{minipage}
    \end{figure}
    \vspace{-1.5em}


    % 教育背景(本科生)
    % 这类\fa开头的都是font awesome里的logo,想换成其他logo的话,可以搜一下fontawsome支持的icon指令。
    \section{\makebox[\widthof{\faGraduationCap}][c]{\color{SWU_Blue}{\faGraduationCap}}\quad 教育背景}

    % 硕士研究生
    {\large \textbf{西南大学}}  学士 \hfill {重庆} \\
    {{人工智能学院}}·人工智能 \hfill {20XX年9月-20XX年6月} \\
    {主修课程}:课程1、课程2、课程3、课程4\ 等
    \hfill {GPA}:4.0/4.0

    % 博士研究生
    \vspace{0.5em}
    {\large \textbf{西南大学}}  硕士 \hfill {重庆} \\
    {{人工智能学院}}·人工智能 \hfill {20XX年9月-20XX年6月} \\
    \textbf{研究方向}:方向1、方向2、方向3、方向4\ 等
    \hfill {GPA}:100/100

    \vspace{0.5em}
    {\large \textbf{西南大学}}  博士 \hfill {重庆} \\
    {{人工智能学院}},导师:\href{导师的个人主页.site}{导师名字}\ 导师职称 \hfill {20XX年9月-至今} \\
    \textbf{研究方向}:方向1、方向2、方向3、方向4\ 等


    % 科研著作(研究生)
    \section{\makebox[\widthof{\faGraduationCap}][c]{\color{SWU_Blue}{\faFlask}}\quad 科研成果}
    
    \vspace{0.5em}
    This is Another Paper. \\
    \textbf{Mingzi Nide}, Shidi Nide, Daoshi Nide. \hfill 
    在投 \textbf{Conference B} (CCF-B类会议)

    \vspace{0.5em}
    This is A Journal Paper.\\
    \textbf{Mingzi Nide}, Shixiong Nide, Daoshi Nide. \hfill 
    在投 \textbf{Journal C} (SCI-2区)


    % 项目经历\科研经历\项目与教学(标题请根据需要修改)
    \section{\makebox[\widthof{\faChalkboardTeacher}][c]{\color{SWU_Blue}{\faChalkboardTeacher}}\quad 横向项目}
    % 项目模板1
    \vspace{0.5em}
    {\large{\textbf{项目名称}}} \hfill {横向/纵向项目-已完结/进行中}\\
    \textbf{你在项目中扮演的角色} \hfill 20XX年9月至20XX年9月\\
    项目简介:XXX,XXX。
    
    % 项目模板2
    \vspace{0.5em}
    {\large{\textbf{国企单位-基于视觉识别算法改进的厂区安全检测平台}}} 项目负责人 \hfill {20XX年10月至20XX年1月}\\
    主要内容:数据集制作;模型复现训练;核心算法实现;GUI开发;专利撰写\ 等

    % 技能特长(标题根据个人需求修改)
    \section{\makebox[\widthof{\faWrench}][c]{\color{SWU_Blue}{\faWrench}}\quad 技能荣誉 }
    \vspace{0.5em}
    \begin{itemize}
        \item 英语:六级800分、托福200分
        \item 编程:Python, R, MATLAB, C
        \item 证书:CET-6, 教师资格证书
        \item 其他: 专利X项, 软著X项,其他奖项若干
    \end{itemize}
    

    %%%%%%%%%  3、新加一页的模板,根据需要可添加  %%%%%%%%%%%
    % \newpage

    % 页眉页脚不要删。
    % % 页眉:校标组合+学院名
    % \begin{tikzpicture}[remember picture, overlay]
    %     \node[anchor=north, inner sep=0pt](header) at (current page.north){
    %         \includegraphics[width=\paperwidth]{images/header.png}
    %     };
    %     \node[anchor=west](school_logo) at (header.west){
    %         \hspace{0.5cm}
    %         \includegraphics[width=0.15\textwidth]{images/swu_logo_1_AIenhanced.png}
    %     };
    %     \node[anchor=east](school_name) at(header.east){
    %         \textcolor{white}{\textbf{\school}}
    %         \hspace{0.5cm}
    %     };
    % \end{tikzpicture}
    % \vspace{-4em}
    
    % % 页脚,联系方式
    % \begin{tikzpicture}[remember picture, overlay]
    %     \node[anchor = south, inner sep=0pt] at (current page.south){
    %         \includegraphics[width=\paperwidth]{images/footer.png}
    %     };
    %     % 联系方式
    %     \node[anchor=center] at(footer.center){\contact};
    % \end{tikzpicture}
    
    % % 背景
    % \begin{tikzpicture}[remember picture, overlay]
    %     \node[opacity=0.1] at(current page.center){
    %         \includegraphics[width=0.7\paperwidth, keepaspectratio]{images/swu_logo_big_blue.png}
    %     };
    % \end{tikzpicture}


    % % 竞赛经历
    % \section{\makebox[\widthof{\faTrophy}][c]{\color{SWU_Blue}{\faTrophy}}\quad 竞赛经历}
    % \vspace{-1em}
    % \begin{table}[h!]
    %     \begin{tabularx}{\textwidth}{Xp{\widthof{第零负责人}}p{\widthof{国家级-第100名}}p{\widthof{2030年13月}}}
    %         \textbf{比赛1} & 第一负责人 & 国家级-第10名 & 2023年4月 \\
    %         \textbf{比赛2} & 个人参赛 & 国家级-一等奖 & 2023年8月\\
    %         \textbf{比赛3} & 个人参赛 & 省级-一等奖 & 2022年12月\\
    %         % 同理,可以自己加
    %     \end{tabularx}
    % \end{table}

    % % 技能特长
    % \section{\makebox[\widthof{\faWrench}][c]{\color{SWU_Blue}{\faWrench}}\quad 技能特长}
    % \vspace{0.5em}
    % \begin{itemize}
    %     \item 熟练使用\Cpp 、Python、Matlab编程语言。
    %     \item 熟悉Windows与Linux端开发。
    %     \item 熟练使用Tensorflow,Pytorch等深度学习框架。
    %     \item 熟练掌握\Cpp 与Python环境下OpenCV与Qt应用的开发,且熟练使用Qt Creator软件。
    %     \item 熟练使用Altium Designer与LCEDA进行封装绘制与板子设计。
    %     \item 熟练使用Keil,Arduino IDE等集成开发软件。
    %     \item 了解模式识别,强化学习,遗传算法,知识蒸馏等相关概念。
    % \end{itemize}

    % % 所获荣誉
    % \section{\makebox[\widthof{\faStar}][c]{\color{SWU_Blue}{\faStar}}\quad 所获荣誉}
    % \vspace{-1em}
    % \begin{multicols}{2}
    %     \begin{itemize}
    %         \item 某年学业先进个人
    %         \item 某年某奖学金某等奖
    %         \item 某大使
    %         \item 某年某奖学金某等奖
    %         \item 某年优秀团员称号
    %         \item 某年某称号
    %     \end{itemize}
    % \end{multicols}

    % % 其他
    % \section{\makebox[\widthof{\faInfo}][c]{\color{SWU_Blue}{\faInfo}}\quad 其他}
    % \begin{itemize}
    %     \item 英语水平-CET6级xxx分
    %     \item 计算机几级证书
    %     \item xx几级证书
    %     \item 技术博客: 某网址
    %     \item 教师资格证:xxx
    %     \item 普通话证书:几级几等
    %     \item 文字排版:\LaTeX
    % \end{itemize}

\end{document}
